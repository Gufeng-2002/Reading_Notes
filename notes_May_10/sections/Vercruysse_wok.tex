\section{Vercruysse's work (2022)}
The following notes summarize the work of Vercruysse (2022) on the environmental gradients in saline fens of Alberta.
 The study focuses on the relationship between morphometry and water chemistry in saline wetlands, 
 particularly in relation to the distribution of aquatic vegetation.

\subsection{Data Source (Chapter 2)}
Data were collected from 52 waterbodies in a saline fen complex in Alberta during 2020, categorized into three morphometry types: flark (n = 38), flark/pond (n = 9), and pond (n = 5).

\textbf{Environmental variables collected:}
\begin{itemize}
  \item \textbf{Water quality:} Temperature ($^\circ$C), pH, dissolved oxygen (mg/L), specific conductance ($\mu$S/cm), redox potential (mV)
  \item \textbf{Nutrients and ions:} Chloride, sulfate (SO$_4$-S), phosphate (PO$_4$-P), ammonium (NH$_4$-N), total organic nitrogen (TON-N)
  \item \textbf{Cations and elements:} Calcium (Ca), Magnesium (Mg), Sodium (Na), Potassium (K), Aluminum (Al), Boron (B), Barium (Ba), Iron (Fe), Manganese (Mn), Lithium (Li), Strontium (Sr), Silicon (Si), Sulfur (S)
  \item \textbf{Physical site info:} Maximum depth (cm), northing and easting coordinates
\end{itemize}

All chemical analyses were conducted at the NRAL facility at the University of Alberta.

\subsection{Analysis Methods (Chapter 2)}

\begin{itemize}
  \item \textbf{Dixon’s Q-test:} Used to identify outlier sites in the environmental dataset, which were removed before multivariate analysis.\\
  \textit{Variables used:} All environmental variables listed below were included:
  \begin{itemize}
    \item pH, temperature, dissolved oxygen, specific conductance, redox potential
    \item Nutrients: phosphate (PO$_4$), ammonium (NH$_4$), nitrate (NO$_3$), total organic nitrogen (TON)
    \item Major ions and elements: Na, K, Ca, Mg, Cl, SO$_4$, Al, Fe, Mn, Sr, Si, Li, B, Ba, S
  \end{itemize}

  \item \textbf{Pearson’s correlation:} Tested relationships between morphometry class (coded as flark = 1, flark/pond = 2, pond = 3) and each environmental variable.\\
  \textit{Variables used:} Same as above (individual environmental variables vs. morphometry index).

  \item \textbf{Holm’s correction:} Applied to adjust p-values for multiple comparisons in the Pearson correlation analysis.\\
  \textit{Variables used:} Adjusted p-values from all Pearson tests involving morphometry and environmental variables.

  \item \textbf{Principal Component Analysis (PCA):}
  \begin{itemize}
    \item Conducted on water chemistry and environmental variables to reduce dimensionality and detect structure in gradients.
    \item Morphometry class was excluded as a variable to avoid bias.
    \item Varimax rotation was used to enhance interpretability of component loadings.
  \end{itemize}
  \textit{Variables used:} Full set of chemical and environmental variables (same as above), excluding morphometry.
\end{itemize}

\subsection{Findings (Chapter 2)}
\begin{itemize}
  \item \textbf{Depth} was the only variable moderately correlated with morphometry (Pearson’s r = 0.512), though it was not statistically significant after Holm correction.
  \item Several variables had weak correlations (r = 0.3–0.5) with morphometry, including: specific conductance, chloride, sulfate, sodium, calcium, magnesium, sulfur, strontium, boron, and northing.
  \item Phosphate concentration showed the most notable difference in means among morphometry types, highest in flark/ponds and lowest in ponds.
  \item PCA revealed no clear grouping by morphometry — suggesting overlapping environmental conditions across wetland types
\end{itemize}

\subsection{Data Source (Chapter 3)}
Samples were collected from 52 waterbodies in a boreal saline fen in Alberta between September 6–8, 2020. Wetlands were categorized into three morphometry types:
\begin{itemize}
  \item Flark (n = 38)
  \item Flark/pond (n = 9)
  \item Pond (n = 5)
\end{itemize}

\textbf{Biological data:}
\begin{itemize}
  \item Aquatic invertebrates sampled using CABIN protocol (D-frame net, 250~$\mu$m mesh).
  \item 20 jabs per site, with organisms sieved through 4.00, 1.00, 0.50, and 0.25~mm mesh sizes.
  \item Two replicate samples per site were preserved in ethanol and identified to the lowest taxonomic level.
\end{itemize}

\textbf{Water chemistry data:}
\begin{itemize}
  \item Parameters included: specific conductance, pH, dissolved oxygen, temperature, redox potential.
  \item Ion and nutrient concentrations: Na, K, Ca, Mg, Cl, SO$_4$, PO$_4$, NH$_4$, TON, Fe, Mn, Sr, etc.
\end{itemize}

\subsection{Analysis Methods (Chapter 3)}
\begin{itemize}
  \item \textbf{Community Composition Assessment:}
    \begin{itemize}
      \item Calculated total invertebrate abundance and family richness from site replicates.
    \end{itemize}

  \item \textbf{Non-metric Multidimensional Scaling (NMDS):}
    \begin{itemize}
      \item Applied to genus-level community composition.
      \item Hellinger transformation used on abundance data.
      \item Bray-Curtis dissimilarity used for ordination.
      \item Environmental vectors (e.g., conductance, depth, pH) fitted to NMDS axes.
    \end{itemize}

  \item \textbf{Indicator Species Analysis (IndVal):}
    \begin{itemize}
      \item Tested for taxa associated with wetland types (flark, flark/pond, pond).
      \item No taxa showed statistically significant indicator values.
    \end{itemize}

  \item \textbf{Linear Regression:}
    \begin{itemize}
      \item Regressed total invertebrate abundance and family richness against log-transformed specific conductance.
      \item Tested the hypothesis that richness decreases and abundance increases with salinity.
    \end{itemize}
\end{itemize}

\subsection{Findings (Chapter 3)}
\begin{itemize}
  \item Community composition did not differ significantly among morphometry types.
  \item NMDS showed overlapping site groupings; no distinct clustering by wetland type.
  \item Indicator species analysis revealed no statistically significant indicator taxa.
  \item Depth was the only variable marginally associated with morphometry.
  \item Conductance and major ion concentrations showed weak, non-significant relationships with wetland type.
  \item Regression analyses did not strongly support hypothesized trends in richness or abundance relative to conductivity.
\end{itemize}

